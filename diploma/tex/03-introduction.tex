\chapter*{ВВЕДЕНИЕ}
\addcontentsline{toc}{chapter}{ВВЕДЕНИЕ}

ДР является средством воздействия информационных технологий на окружающую среду \cite{osti2019real}. Она позволяет совместить искусственно сгенерированное изображение с изображением реального мира с помощью различных датчиков, получающих информацию об окружающем мире, и специального ПО. Технология ДР предоставляет возможность накладывать элементы виртуальной информации поверх изображений объектов физического мира в реальном времени. Она используется во множестве сфер деятельности человека: медицина, построение анатомических моделей, образование, туризм и других \cite{tech-ar}.

Важными условиями реалистичного восприятия виртуальных объектов, вписанных в изображение реального мира, является моделирование свойств текстур и оптических свойств поверхности (отражение, пропускание, преломление света) и способности виртуальных объектов отбрасывать тени в соответствии с условиями освещения. Виртуальные объекты должны визуализироваться таким образом, чтобы поведение виртуальных теней, отбрасываемых виртуальными объектами, соответствовало поведению теней от реальных объектов и не вызывало у пользователя дискомфорта при наблюдении смешанного изображения \cite{bogdanov}. 

Для моделирования свойств текстур и оптических свойств поверхности используются изображения высокого разрешения (текстуры), карты нормалей, карты рельефа, карты отражения и т. п. Однако даже при их использовании, если виртуальный объект не отбрасывает тень, это не решает проблему реалистичного восприятия виртуальных объектов, т. к. виртуальный объект будет выделяться на фоне окружения засчет отсутствия взаимодействия с окружающим светом, тем самым вызывать дискомфорт у пользователя. Таким образом, согласование систем освещения реального и виртуального миров является основополагающей \cite{osti2019real}.

В наложении теней учавствуют 3 сущности: ИС, объект, отбрасывающий тень, и объект, на который отбрасывается тень. Соответственно, возможны комбинации:

\begin{itemize}
	\item[---] ИС может быть виртуальным или реальным;
	\item[---] объект, отбрасывающий тень, может быть виртуальным или реальным;
	\item[---] объект, на который отбрасывается тень, может быть виртуальным или реальным;
\end{itemize}

В данной работе рассматривается случай, при котором ИС является реальным, объект, отбрасывающий тень, является виртуальным и объект, на который отбрасывается тень, является реальным.

\textbf{Цель работы} -- исследование методов наложения теней в ДР на основе информации о глубине точек кадра и разработка собственного метода наложения теней.

Для достижения поставленной цели нужно решить следующие задачи:

\begin{itemize}
	\item[---] провести анализ предметной области наложения теней;
	\item[---] провести обзор существующих методов наложения теней в ДР на основе информации о глубине точек кадра и привести результаты сравнительного анализа;
	\item[---] разработать и описать собственный метод наложения теней в ДР на основе информации о глубине точек кадра;
	\item[---] разработать программное обеспечение, реализующее описанный метод, и выполнить его тестирование;
	\item[---] провести исследование результатов разработанного метода при проецировании теней от виртуального объекта на различные поверхности;
	\item[---] выполнить сравнение результатов работы реализованного метода с результатами, полученными с помощью существующих аналогов.
\end{itemize}