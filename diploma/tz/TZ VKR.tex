\documentclass[a4paper]{bmstu}

\begin{document}
	\noindent Тема: \textit{<<Метод наложения теней в дополненной реальности на основе информации о глубине точек кадра>>.}
	
	""\newline\noindent Задание:
	
	\begin{enumerate}
		\item \textbf{Аналитический раздел}
		
		Провести анализ предметной области наложения теней в дополненной реальности. Рассмотреть методы наложения теней на основе информации о глубине точек кадра. Провести обзор существующих методов и привести результаты сравнительного анализа. Формализовать постановку задачи в виде IDEF0-диаграммы.
		
		\item \textbf{Конструкторский раздел}
		
		Разработать метод наложения теней в дополненной реальности на основе информации о глубине точек кадра. Изложить особенности разрабатываемого метода. Сформулировать и описать ключевые шаги метода в виде схем алгоритмов. Описать структуры данных, используемые в алгоритмах. Описать взаимодействие отдельных частей системы.
		
		\item \textbf{Технологический раздел}
		
		Обосновать выбор языка и средств программной реализации метода наложения теней в дополненной реальности. Разработать программное обеспечение, реализующее описанный метод, и выполнить его тестирование. Описать формат входных и выходных данных и взаимодействие пользователя с программным обеспечением.
		
		\item \textbf{Исследовательский раздел}
		
		Провести исследование результатов разработанного метода при проецировании теней от виртуального объекта на различные поверхности. Выполнить сравнение результатов работы реализованного метода наложения теней в дополненной реальности с результатами, полученными с помощью существующих аналогов.
		
	\end{enumerate}
\end{document}
