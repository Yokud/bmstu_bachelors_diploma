\chapter*{РЕФЕРАТ}

Объем РПЗ составляет \begin{NoHyper}\pageref{LastPage}\end{NoHyper} страницу, содержит \totfig~иллюстрации, \tottab~таблицу, 1 приложение и 26 использованных источников.

\textit{Ключевые слова}: дополненная реальность, компьютерное зрение, машинная графика, наложение теней, изображение высокого динамического диапазона, RGBD-изображение, восстановление параметров освещения.

В работе были сформулированы цель и задачи работы, изучена предметная область, обоснована актуальность задачи, описаны основные определения дополненной реальности, компьютерного зрения, моделей освещения и наложения теней. Также был проведен обзор существующих методов получения данных о глубине окружения, моделей освещения и способов построения теней, а также были определены основные сложности решения задачи наложения теней в дополненной реальности. Помимо этого был проведен обзор существующих методов наложения теней в дополненной реальности, выбраны критерии сравнения и проведена классификация существующих методов наложения теней в дополненной реальности.