\chapter{Анализ предметной области}

\section{Предметная область}

Предметная область -- наложение теней  на виртуальные объекты в ДР на основе информации о реальном окружении.

Объект исследования -- методы наложения теней  на виртуальные объекты в ДР.

Предмет исследования -- информация, требуемая для восстановления модели освещения окружения.

\section{Актуальность задачи}

Рендеринг синтетических объектов в реальных сценах является важным применением компьютерной графики, особенно в области архитектуры и визуальных эффектов. Часто предмет обстановки, реквизит, цифровое существо или актер должны быть плавно перенесены в реальную сцену. Эта сложная задача требует, чтобы объекты освещались последовательно с поверхностями, находящимися поблизости от них, и чтобы взаимодействие света между объектами и их окружением было должным образом смоделировано. В частности, объекты должны отбрасывать тени, появляться в отражениях и преломлять, фокусировать и излучать свет точно так же, как это делали бы реальные объекты \cite{debevec2008rendering}.

Проблема визуализации виртуальных объектов в реальном мире -- низкое качество конечного изображения, из-за чего у пользователя не создается ощущение погружения в происходящее. Проблему низкого качества изображения можно разделить на две части: проблема материала и проблема освещения. Однако даже с использованием ультрареалистичных материалов для виртуальных объектов наблюдение за ними без использования системы освещения и отбрасывания теней не способствует реалистичности сцены \cite{osti2019real}.

Таким образом, использование системы освещения, в частности, наложение теней на виртуальные объекты в ДР имеет важное значение для высокого качества конечного изображения, поскольку это уменьшает дискомфорт от восприятия виртуальных объектов в реальном окружении и увеличивает уровень ощущения погружения в происходящее.

\section{Основные определения}

\subsection{Дополненная реальность}

ДР -- технология интеграции цифровой информации в виде изображений, компьютерной графики, текста, видео, аудио и т.д. и объектов действительного (физического) мира в режиме реального времени \cite{tech-ar}.



\subsection{Данные об окружении}



\subsection{Восстановление модели освещения}



\subsection{Наложение теней}



\section{Специфика задач наложения теней в ДР}



\section{Сложность наложения теней в ДР}


