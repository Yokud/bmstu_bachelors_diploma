\chapter{Аналитическая часть}


\section{Обзор существующих методов наложения теней в ДР}

\subsection{Метод с использованием HDR изображений}



\subsection{Метод с использованием RGBD камеры}



\subsection{ARPAS}

ARPAS (Augmented Reality Photorealistic Ambient Shadows) -- метод наложения теней, который в качестве исходных данных использует информацию о глобальном освещении окружающей среды и любых источниках света, присутствующих вокруг пользователя, из HDR и LDR изображений \cite{rtsm}. Исходные изображения должны обладать следующими свойствами:

\begin{itemize}
	\item они должны быть всенаправленными, т.е. для каждого направления пространства имеется пиксель, представляющий это направление;
	\item значения пикселей соответствуют количеству света, поступающего с этого направления.
\end{itemize}

ARPAS состоит из 4 основных этапов:

\begin{enumerate}
	\item захват изображения;
	\item обработка изображения;
	\item поиск положения источников света;
	\item размещение источников света на виртуальной сцене.
\end{enumerate}

\subsubsection*{Захват изображения}

На основе исходных снимков создается т. н. карта сферы, которая представляет собой сферическое изображение на 360 градусов окружения, где будут размещены синтетические объекты. Чтобы удовлетворить свойство всенаправленности, наиболее часто используемым методом является фотографирование зеркальной сферы: этот метод позволяет получать свет, исходящий из-за сферы, поскольку лучи за сферой отклоняются и захватываются камерой спереди. Более простой метод состоит в том, чтобы сделать несколько фотографий всего окружения и скомпоновать их вместе, накладывая друг на друга, чтобы сформировать карту сферы \cite{rtsm}.

\subsubsection*{Обработка изображения}

Полученную карту сферы конвертируют из RGB-изображения в черно-белое для более простого применения порогового значения к
значениям цвета пикселей, поскольку они варьируются только от 0 (или больше для слишком ярких изображений) до 255 (или меньше для слишком темных изображений). Стоит отметить, что не все RGB-изображения после конвертировании в черно-белый формат имеют пиксели, которые варьируются от 0 до 255, например, слишком яркие или слишком темные.

Далее проверяется следующее условие:

\begin{equation}
	\frac{Max(PixelValue)}{Average(PixelValue)} \geq 1.5
\end{equation}

Если это условие верно, то это означает, что разница между максимальным значением и средним значением пикселей достаточно, чтобы утверждать, что существует видимая разница между окружающим светом и возможным точечным светом \cite{rtsm}.

\subsubsection*{Поиск положения источников света}



\subsubsection*{Размещение источников света на виртуальной сцене}



\subsection{ARShadowGAN}



\subsection{DepthLab}



\section{Анализ предметной области}



\section{Критерии сравнения}



\section{Классификация существующих методов}