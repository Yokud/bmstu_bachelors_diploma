\chapter{Аналитическая часть}


\section{Обзор существующих методов наложения теней в AR}

\subsection{Метод с использованием HDR изображений}

Метод основан на анализе HDR-изображений \cite{hdri}, восстанавливая параметры освещения в сложных сценах с несколькими источниками света. В качестве исходных данных метод использует HDR-изображение видимого пространства и карту глубины сцены.

На основе полученных данных строится трехмерная модель окружения, после чего происходит поиск теней и распознавание объектов, которые их отбрасывают. 

Схема метода представлена на рисунке \ref{img:HDR_Method}.

\includeimage
	{HDR_Method}
	{f}
	{H}
	{0.25\textwidth}
	{Схема метода наложения теней в AR с использованием HDR изображений}

\subsection{Метод с использованием RGBD-камеры}

\subsection{Метод с использованием сверточных нейронных сетей}

\subsection{Метод с использованием сверточных нейронных сетей и трассировки теневых лучей}

\section{Анализ предметной области}



\section{Критерии сравнения}



\section{Классификация существующих методов}