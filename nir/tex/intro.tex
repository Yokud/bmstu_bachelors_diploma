\chapter*{ВВЕДЕНИЕ}
\addcontentsline{toc}{chapter}{ВВЕДЕНИЕ}

ДР способна внедрять виртуальные объекты в среду реального мира. Она используется во множестве сфер деятельности человека: медицина, построение анатомических моделей, образование, туризм и т. д. \cite{tech-ar}.

Проблема визуализации виртуальных объектов в реальном мире -- низкое качество изображения, из-за чего у пользователя не создается ощущение погружения в происходящее. Проблему низкого качества изображения можно разделить на две части: проблема материала и проблема освещения. Однако даже с использованием ультрареалистичных материалов для виртуальных объектов наблюдение за ними без использования системы освещения и отбрасывания теней не способствует реалистичности сцены \cite{osti2019real}.

Таким образом, использование системы освещения, в частности, наложение теней на виртуальные объекты в ДР имеет важное значение для высокого качества конечного изображения.

Цель работы -- классифицировать методы наложения теней в дополненной реальности.

Для достижения поставленной цели нужно решить следующие задачи:

\begin{itemize}
	\item провести обзор существующих методов наложения теней в ДР;
	\item провести анализ предметной области наложения теней в ДР;
	\item сформулировать критерии сравнения методов;
	\item классифицировать существующие методы.
\end{itemize}