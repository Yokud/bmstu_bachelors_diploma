\chapter*{ВВЕДЕНИЕ}
\addcontentsline{toc}{chapter}{ВВЕДЕНИЕ}

ДР является средством воздействия информационных технологий на окружающую среду \cite{osti2019real}. Она позволяет совместить искусственно сгенерированное изображение с изображением реального мира с помощью различных датчиков, получающих информацию об окружающем мире, и специального ПО. Технология ДР предоставляет возможность накладывать элементы виртуальной информации поверх изображений объектов физического мира в реальном времени. Часть виртуальной информации можно представить в виде текстовой информации, графики, звуков, и других. Она используется во множестве сфер деятельности человека: медицина, построение анатомических моделей, образование, туризм и других \cite{tech-ar}.

Важными условиями реалистичного восприятия виртуальных объектов, вписанных в изображение реального мира, является моделирование свойств текстур и оптических свойств поверхности (отражение, пропускание, преломление света) и способности виртуальных объектов отбрасывать тени в соответствии с условиями освещения. Виртуальные объекты должны визуализироваться таким образом, чтобы поведение виртуальных теней, отбрасываемых виртуальными объектами, соответствовало поведению теней от реальных объектов и не вызывало у пользователя дискомфорта при наблюдении смешанного изображения \cite{bogdanov}.

В наложении теней учавствуют 3 сущности: ИС, объект, отбрасывающий тень, и объект, на который отбрасывается тень. Соответственно, возможны комбинации:

\begin{itemize}
	\item ИС может быть виртуальным или реальным;
	\item объект, отбрасывающий тень, может быть виртуальным или реальным;
	\item объект, на который отбрасывается тень, может быть виртуальным или реальным;
\end{itemize}

В данной работе рассматривается следующий случай:

\begin{itemize}
	\item ИС является реальным;
	\item объект, отбрасывающий тень, является виртуальным;
	\item объект, на который отбрасывается тень, является реальным;
\end{itemize}

\textbf{Цель работы} -- провести анализ предметной области и классифицировать методы наложения теней в ДР.

Для достижения поставленной цели нужно решить следующие задачи:

\begin{itemize}
	\item провести анализ предметной области наложения теней;
	\item провести обзор существующих методов наложения теней в ДР;
	\item сформулировать критерии сравнения методов;
	\item классифицировать существующие методы.
\end{itemize}