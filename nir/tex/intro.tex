\chapter*{ВВЕДЕНИЕ}
\addcontentsline{toc}{chapter}{ВВЕДЕНИЕ}

ДР является средством воздействия информационных технологий на окружающую среду. Они позволяют создать в реальности дополнительные цифровые элементы или видоизменить органы чувств пользователя с помощью специального устройства. Интерактивная технология предоставляет возможность накладывать элементы виртуальной информации поверх объектов физического мира в реальном времени. Часть виртуальной информации можно представить в виде текстовой информации, графики, звуков, реакции на прикосновения и т.д. Также ДР способна внедрять виртуальные объекты в среду реального мира. Она используется во множестве сфер деятельности человека: медицина, построение анатомических моделей, образование, туризм и т. д. \cite{tech-ar}.

Важными условиями реалистичного восприятия виртуальных объектов является наличие соответствующих текстур и оптических свойств поверхности (отражение, пропускание, преломление) и их способность отбрасывать тени в соответствии с условиями освещения. Виртуальные объекты должны визуализироваться таким образом, чтобы их виртуальные тени коррелировали с тенями от реальных объектов и не вызывали чувства дискомфорта при наблюдении смешанного изображения \cite{bogdanov}.

\textbf{Цель работы} -- провести анализ предметной области и классифицировать методы наложения теней в ДР.

Для достижения поставленной цели нужно решить следующие \textbf{задачи}:

\begin{itemize}
	\item провести обзор существующих методов наложения теней в ДР;
	\item провести анализ предметной области наложения теней;
	\item сформулировать критерии сравнения методов;
	\item классифицировать существующие методы.
\end{itemize}