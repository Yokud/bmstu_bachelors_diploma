\chapter*{ЗАКЛЮЧЕНИЕ}
\addcontentsline{toc}{chapter}{ЗАКЛЮЧЕНИЕ}

Была изучена предметная область, обоснована актуальность задачи, описаны основные определения ДР, компьютерного зрения, моделей освещения и наложения теней, был проведен обзор существующих методов получения данных о глубине окружения, моделей освещения и способов построения теней, а также определены сложности решения задачи наложения теней в ДР.

Был проведен обзор существующих методов наложения теней в ДР, в частности, на основе анализа гистограммы изображения окружения, на основе анализа контуров теней, на основе построения теневого объема и с использованием сверточных нейронных сетей и трассировки теневых лучей.

Также были введены критерии сравнения методов и проведена их классификация.

Таким образом, цель работы, провести анализ предметной области и классифицировать методы наложения теней в ДР, была достигнута.