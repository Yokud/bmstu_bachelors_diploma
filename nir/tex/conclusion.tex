\chapter*{ЗАКЛЮЧЕНИЕ}
\addcontentsline{toc}{chapter}{ЗАКЛЮЧЕНИЕ}

Была изучена предметная область, обоснована актуальность задачи, описаны основные определения ДР, компьютерного зрения, моделей освещения и наложения теней, был проведен обзор существующих методов получения данных о глубине окружения, моделей освещения и способов построения теней, а также определены сложности решения задачи наложения теней в ДР.

Был проведен обзор существующих методов наложения теней в ДР, в частности, на основе анализа гистограммы изображения окружения, на основе анализа контуров теней, на основе построения теневого объема и с использованием сверточных нейронных сетей и трассировки теневых лучей.

Также были введены критерии сравнения методов и проведена их классификация.

По результатам сравнения методов можно сказать, что большинство методов, кроме метода на основе построения теневых объемов, предназначено для работы только внутри помещения, но при этом способны распознавать несколько ИС. Также стоит заметить, что метод на основе анализа контуров теней ИС и метод с использованием сверточных нейронных сетей и трассировки теневых лучей соответствуют большинству критериев. Выбор того или иного метода зависит от производительности системы: требования к производительности системы у метода с использованием сверточных нейронных сетей и трассировки теневых лучей выше, чем у метода на основе анализа контуров теней ИС.

Таким образом, цель работы, провести анализ предметной области и классифицировать методы наложения теней в ДР, была достигнута.