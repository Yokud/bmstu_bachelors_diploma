\chapter{Классификация существующих методов}

В данном разделе проводится обзор существующих методов наложения теней в ДР, вводятся критерии сравнения и проводится классификация этих методов.

\section{Обзор существующих методов наложения теней в ДР}

\subsection{Метод на основе анализа гистограммы изображения окружения}

В этом методе в качестве исходных данных используется информация о глобальном освещении окружающей среды и любых источниках света, присутствующих вокруг пользователя, из HDR-изображений \cite{osti2019real}. Исходные изображения должны обладать следующими свойствами:

\begin{itemize}
	\item они должны быть всенаправленными, т.е. для каждого направления пространства имеется пиксель, представляющий это направление;
	\item значения пикселей соответствуют количеству света, поступающего с этого направления.
\end{itemize}

Метод состоит из 4 основных этапов:

\begin{enumerate}
	\item захват изображения;
	\item обработка изображения;
	\item поиск положения источников света;
	\item отрисовка теней.
\end{enumerate}

\subsubsection*{Захват изображения}

На основе исходных снимков создается т. н. карта сферы, которая представляет собой сферическое изображение на 360 градусов окружения, где будут размещены синтетические объекты. Чтобы удовлетворить свойство всенаправленности, наиболее часто используемым методом является фотографирование зеркальной сферы: этот метод позволяет получать свет, исходящий из-за сферы, поскольку лучи за сферой отклоняются и захватываются камерой спереди. Более простой метод состоит в том, чтобы сделать несколько фотографий всего окружения и скомпоновать их вместе, накладывая друг на друга, чтобы сформировать карту сферы \cite{osti2019real}.

\subsubsection*{Обработка изображения}

Полученную карту сферы конвертируют из RGB-изображения в черно-белое для более простого применения порогового значения к
значениям цвета пикселей, поскольку они варьируются только от 0 до 255. Стоит отметить, что не все RGB-изображения после конвертировании в черно-белый формат имеют пиксели, которые варьируются от 0 до 255, например, слишком яркие или слишком темные.

После преобразования изображения можно применить для него пороговое значение, чтобы удалить все пиксели со значением цвета ниже заданного порога. Этот порог выбирается путем анализа гистограммы изображения \cite{img_hists}.

Далее проверяется следующее условие:

\begin{equation}
	\frac{Max(PixelValue)}{Average(PixelValue)} \geq 1.5
\end{equation}

Если это условие верно, то это означает, что разница между максимальным значением и средним значением пикселей достаточно, чтобы утверждать, что существует видимая разница между окружающим светом и возможным точечным светом. В ином случае возможно ошибочное отождествление окружающего света с точечным светом, что приводит, во-первых, к слишком высокой плотности белых пикселей и, во-вторых, к неточному расположению света на этапе поиска положения источников света \cite{osti2019real}.

Далее происходит оценка порогового значения. Если 93\% спектра (начиная с 0) покрывают не менее 98\% пикселей, то можно пороговать оставшиеся 7\%. Этот шаг позволяет обрезать изображение и рассматривать только те области изображения, которые соответствуют реальным источникам света. Полученный результат обрабатывается медианным фильтром размытия в качестве метода шумоподавления, поскольку возможно наличие некоторых областей со значением пикселей выше порогового значения из-за отражений объектов в окружающей среде \cite{osti2019real}.

\subsubsection*{Поиск положения источников света}

У полученного изображения вычисляются контуры источников света. Из полученных контуров определяются его моменты, из которых можно получить центроиды каждого источника света.

Из координат центроида $(x, y)$ можно получить координату $z$ следующим образом. Координата $x$ пропорциональна повороту вокруг оси $Y$, а координата $y$ пропорциональна повороту вокруг оси $X$, из чего следуют соотношения \cite{osti2019real}:

\begin{equation}
	\begin{split}
		\frac{x}{\theta} = \frac{width}{2\pi} \\
		\frac{y}{\phi} = \frac{height}{\pi},
	\end{split}
\end{equation}

где $\theta, \phi$ -- углы поворота вокруг осей $Y$ и $X$ соответственно, $width, height$ -- ширина и высота исходного изображения окружения. Зная углы поворота вокруг осей, можно вычислить модуль координаты $z$ следующим образом:

\begin{equation}
	\begin{split}
		z_{xz} = x\cos(\theta) \\
		z_{yz} = y\cos(\phi), \\
		|z| = \sqrt{z_{xz} ^ 2 + z_{yz} ^ 2} \\
	\end{split}
\end{equation}

где $z_{xz}$ -- координата $z$ проекции точки положения виртуального ИС на плоскость $XZ$, $z_{yz}$ -- координата $z$ проекции точки положения виртуального ИС на плоскость $YZ$. Знак координаты $z$ можно восстановить следующим образом:

\begin{equation}
	z = |z| \cdot sign(z_{xz} \cdot z_{yz})
\end{equation}

Таким образом, становится известно положение ИС в трехмерном пространстве.

\subsubsection*{Отрисовка теней}

Этот этап ничем особым не отличается по сравнению с остальными: отрисовка теней происходит по тем же алгоритмам, что и в машинной графике. Зная местоположение каждого ИС, расставляются их виртуальные аналоги. В итоге происходит синтез и отображение виртуальной сцены с виртуальным объектом, отбрасывающий тень от <<реальных>> ИС.

\subsubsection*{Преимущества и недостатки}

Преимуществами данного метода являются: 
\begin{itemize}
	\item для наложения тени на виртуальный объект требуется только вычислить тень, падающую от него, т. к. все шаги с рассчетом положения ИС уже были проделаны;
	\item возможность распознавания нескольких ИС.
\end{itemize}

Недостатками данного метода являются:
\begin{itemize}
	\item требуется предварительная подготовка информация об окружении, т. е. динамическая смена окружения не предусмотрена;
	\item корректная отрисовка теней происходит только на плоской поверхности, т. е. нельзя воспроизвести её искажение, т. к. нет информации о поверхности проецирования.
\end{itemize}

\subsection{Метод на основе анализа контура теней ИС}

В качестве исходных данных используются RGBD-изображения.

Из исходных данных вычисляется следующий набор данных: точки, соответствующие границам объектов, и точки, соответствующие границам теней от объектов. Предполагается, что имеется прямое соответствие между объектами и их тенями на RGBD-изображении. Количество найденных точек практически не сказывается на точности, но они должны быть расположены равномерно по контуру изображения объекта и отклонение точек от границы контура не должно быть значительным. Количество точек на контурах объектов и их теней не должно быть меньше 3 \cite{shadow_contours_method}.

Полученные наборы точек используются для нахождения положений ИС следующим образом. 

\begin{enumerate}
	\item Строится набор плоскостей параллельно поверхности земли.
	\item Из сопряженных точек, находящимися на границах тени и объекта, пускают лучи, которые пересекают ранее построенные плоскости, тем самым итеративно получают точки пересечения лучей с плоскостями.
	\item Проверяется характер распределения скоплений этих точек на каждой плоскости. Скопление точек, полученное данным методом, уплотняется при приближении к виртуальной плоскости к источнику света. На определенном шаге итераций скопление точек перестанет уплотняться и в дальнейшем будет расширяться. Максимальная плотность скопления свидетельствует о том, что ИС находится на высоте этой виртуальной плоскости.
	\item После этого определяется центры скоплений этих точек. Зная высоту найденной виртуальной плоскости и центры скоплений, можно восстановить положения ИС.	При этом абсолютная ошибка определения координаты ИС по высоте не превышает половины шага, с которым чередуются виртуальные плоскости  \cite{shadow_contours_method}.
	\item Далее уточняются координаты ИС с помощью билинейной интерполяции между двумя виртуальными плоскостями с максимальной плотностью точек.
\end{enumerate}

Также стоит упомянуть случай, когда не был найден ни один ИС. Это может происходить, например, если объекты и их тени были некорректно согласованы или видимый край объекта не является краем, создающим тень \cite{shadow_contours_method}.

Т. к. ограничение данного метода является невозможность восстановить интенсивность и диаграмму излучения ИС, то тип найденного ИС устанавливается точечным, а диаграмма излучения -- Ламбертовой \cite{shadow_contours_method}.

\subsubsection*{Преимущества и недостатки}

Преимуществами данного метода являются:

\begin{itemize}
	\item работа в помещении;
	\item возможность распознать несколько ИС. 
\end{itemize}

Недостатками данного метода являются:

\begin{itemize}
	\item зависимость качества его работы от результатов распознавания контуров объектов и теней. Устройство распознавания может распознать только часть контуров, например, распознанная часть тени может не соответствовать обнаруженному контуру объекта.
\end{itemize}

\subsection{Метод на основе построения теневых объемов}

Данный метод состоит из трех этапов.

\begin{enumerate}
	\item Обнаружение теней из видео в реальном времени.
	\item Построение теневого объема.
	\item Отрисовка виртуального объекта и синтез изображения.
\end{enumerate}

В методе текущий кадр, т. е. $k$-й кадр, разделяется на две области: область проекции и новую область. 

Предыдущий кадр, т. е. $k-1$-й кадр, проецируется на $k$-й кадр с оцененной позицией камеры и информацией о глубине точек кадра. Область, которая являестя результатом пересечения двух кадров, называется областью проекции. Новая область -- область $k$-го кадра, не попавшая в область проекции, т. е. разность между областью проекции и $k$-м кадром. Далее фильтруются данные глубины кадра от шумов и оценивается положение камеры. Затем происходит обнаружение граней новых теней в новой области и добавляются к старым. 

На основе обнаруженных граней теней строится теневой объем и уточняется с помощью адаптивной стратегии выборки \cite{THOMASIAN2022385} для достижения плавных эффектов отбрасывания теней.

В итоге происходит синтез и отображение конечного изображения с виртуальным объектом.

\subsubsection*{Обнаружение теней из видео в реальном времени}

Поиск теней происходит в пространстве изображения. 

Большая часть теней распределена по всей области проекции. В этой области грани тени кадра $k$ могут быть непосредственно спроецированы из кадра $k - 1$. Чтобы устранить ошибки проецирования, вызванные зашумленными параметрами камеры и данными о глубине, происходит оптимизация первоначально спроецированных граней тени в области проецирования. Грани тени в новой области обнаруживаются и объединяются с гранями теней из области проекции путем обеспечения согласованности направления градиента края \cite{wei2019simulating}.

""\newline
\indent\textbf{Область проекции}

Для оптимизации контуров теней используется алгоритм Кэнни. К полученным результатам применяют фильтрацию среднего сдвига чтобы сгладить контуры теней и избежать влияния мягких теней и шума изображения. Затем пиксель на грани тени, для которого отношение интенсивности двух сторон его направления градиента превышает заданный порог (0.65 от темной стороны к светлой стороне), идентифицируется как находящийся на краю тени. Далее происходит объединение пикселей изначального контура теней с пикселями контуров теней, полученные после оптимизации, путем поиска пиекселей граней теней в небольшой окрестности (3 на 3 пикселя) каждого пикселя изначального контура теней. Если в окрестности появляется пиксель контура тени после оптимизации, то он заменяет пиксель изначального контура тени \cite{wei2019simulating}.

""\newline
\indent\textbf{Новая область}


""\newline
\indent\textbf{Объединение граней тени}


\subsubsection*{Построение теневого объема}



\subsubsection*{Преимущества и недостатки}

Преимущества:
\begin{itemize}
	\item моделирует систему освещения во внешнем пространстве;
	\item моделирует взаимодействие между тенями реальных и виртуальных объектов.
\end{itemize}


Недостатки:
\begin{itemize}
	\item не позволяет моделировать освещение в помещениях;
	\item работает только со статическими тенями: динамические тени не распознаются, т. к. новые грани тени добавляются к граням предыдущего кадра путем обеспечения согласованности направления градиента края;
	\item плохо работает с мягкими тенями.
\end{itemize}


\subsection{Метод с использованием сверточных нейронных сетей и трассировки теневых лучей}

Суть метода -- определить координаты ИС по теням, отбрасываемые объектами. Он основан на предположении, что для небольших ИС тень объекта является изображением центральной проекции этого объекта на поверхность <<пола>>. Следовательно, зная сопряженные координаты точек границ теней и координаты точек границ объектов, отбрасывающих эти тени, можно восстановить центральную проекцию и найти положение источника света \cite{sns_tras}.

Однако найти точки сопряжения -- задача нетривиальная, особенно для сложных сцен, когда есть много теней от разных ИС и когда тени не проецируются на плоскую поверхность. Метод основан на формировании пучков лучей, исходящих из точек на границе тени. В этом случае предполагается, что среди пучков лучей, испускаемых из тени к объекту, будет хотя бы один, идущий в направлении источника света. Эти лучи формируются из точек, полученных после определения контуров объектов и теней. В качестве контура объекта рассматривается не только его геометрический контур, но и световой контур, т. е. граница света и тени на самом освещаемом объекте. Группа лучей, исходящих из разных точек тени на разные точки объекта, может сформировать каустику, которая будет находиться вблизи источника света. Центр перетяжки этой каустики в пространстве сцены соответствует положению источника света. Поэтому основная задача метода -- найти группу лучей, формирующих каустику \cite{sns_tras}.

В качестве исходных данных используется RGBD-изображение сцены, не требующее калибровки по реальным значениям яркости.

Данный метод состоит из двух этапов.
\begin{enumerate}
	\item Обучение сверточной нейронной сети для определения границ объектов и теневых областей RGBD-изображений, полученных устройством ДР.
	\item Использование алгоритмов машинного зрения для определения положения источников освещения в сцене.
\end{enumerate}

Более подробно метод выглядит так.

\begin{enumerate}
	\item Определяются все теневые области на изображении.
	\item Идентифицируются объекты, отбрасывающие тени, и определяются границы объектов, включая световые границы в области освещаемой и теневой части объекта. Точки этих границ формируются и сохраняются.
	\item Формируется облака точек вероятного пересечения лучей, исходящих из разных точек тени и объекта. Образуются пары несопряженных лучей, т. е. лучи должны исходить из разных точек через разные точки одного объекта. Поскольку фактическое пересечение таких лучей невозможно, выполняется поиск точки на отрезке с минимальным расстоянием, соединяющим две эти прямые. Точки позади объекта или за пределами области определения сцены отбрасываются. 
	\item Точки, полученные в результате пересечения траекторий лучей, помечаются номером объекта, через который прошел луч. Эта маркировка позволяет сортировать сформированные лучи.
	\item Происходит анализ областей скоплений точек, которые принимаются за положение источников света. Для каждой области координаты ИС усредняются, и средняя точка берется за точку положения ИС.
	\item Для найденных точек проверяется правильность нахождения координат ИС. Для этого от источника света на границе тени испускаются лучи и оценивается отклонение координат соответствующих точек от ближайших точек границ объекта. Если отклонение находится в пределах допуска, то найденная точка принимается за центральную точку ИС, в противном случае источник света считается ложным и отклоняется. Кроме того, близкорасположенные ИС, найденные для различных объектов, объединяются в один источник света \cite{sns_tras}.
\end{enumerate}

\subsubsection*{Определение контуров теней и объектов}

Изображения в оттенках серого и цветные изображения могут содержать значительный шум, заключающийся в случайных вариациях яркости или цветов точек изображения. Поэтому для определения контуров объектов и теней необходимо сперва устранить шум изображения, для чего используются различные методы фильтрации и алгоритмы компьютерного зрения. Для этого используются алгоритмы Кэнни \cite{canedgedetect} для обнаружения границ изображения, затем размытие по Гауссу \cite{gaus_smooth} и операция наращивания \cite{dilation} для устранения шума на границах изображения. Чтобы оставить только контуры границ, используется алгоритм скелетизации \cite{skeleton}, который уменьшает бинарные объекты до ширины одной точки изображения. 

После определения всех контуров объектов и теней на изображении необходимо найти соответствие между ними. В первую очередь строятся регионы интересов \cite{roi} области контуров, и если они соприкасаются, т. е. имеют общие границы, то с большой вероятностью контур тени соответствует контуру объекта \cite{sns_tras}.

Кроме того, используется еще один метод сопоставления контуров, заключающийся в использовании функции, вычисляющей и сравнивающей по заданным регионам интересов <<моменты>> контуров изображений объектов и теней сцены. Моменты изображения представляют собой средневзвешенное значение интенсивности пикселей изображения, т. е. это суммарная характеристика контура, рассчитанная интегрированием (суммированием) всех пикселей контура. Все что необходимо -- это вычислить сумму интенсивностей всех пикселей и получить на выходе значение. Далее в функцию сравнения контуров подаются полученные значения и возвращается метрика, показывающая сходство. Чем ниже результат на выходе функции (чем ближе она к нулю), тем больше соответствие и тем вероятнее, что сравниваемые контуры тени и объекта имеют одно происхождение, т. е. тень была сформирована данным объектом \cite{sns_tras}.

\subsubsection*{Формирование лучей}

После того, как были определены все необходимые координаты на исходном изображении контуров объектов и их теней, начинается процесс формирования лучей. Они формируются с заданным шагом по контуру, например, исходя из соображения, что на контуре изображения и тени не должно быть больше 10 или 20 точек \cite{sns_tras}. Исходя из этого на контурах выбираются точки с соответствующим шагом, через которые затем выпускаются лучи, и вычисляются точки, находящиеся на минимальном расстоянии между этими лучами, т. е. точки перетяжки лучей. Вычисление точек перетяжки основывается на методе наименьших квадратов, что является стандартным подходом в регрессионном анализе для аппроксимации решения переопределенных систем путем минимизации суммы квадратов, полученных в результатах каждого отдельного уравнения \cite{mnk}.

Далее определяется максимальная плотность точек перетяжки. По найденным точкам в области наибольшей плотности вычисляются моменты и находится средняя точка.

Необходимо отметить, что если в процессе поиска координат источников света использовались два или более объектов сцены, то найденные облака точек, имеющие максимальную плотность и характеризующие источники света от разных групп объектов – теней, можно объединять в один общий источник света, имеющий конечный размер. Это объединение можно делать только в том случае, если облака точек были порождены различными объектами, поскольку один объект, формирующий разные тени, не может создать один источник \cite{sns_tras}.

\subsubsection*{Преимущества и недостатки}

Преимуществами данного метода являются:

\begin{itemize}
	\item высокая точность (почти 94 \% \cite{sns_tras});
	\item возможность распознать сложные тени и несколько ИС.
\end{itemize}

Недостатками данного метода являются:

\begin{itemize}
	\item высокие требования к производительности системы \cite{tras}.
\end{itemize}

\section{Критерии сравнения}

восстановление нескольких ИС

работа метода в помещении

работа метода под открытым небом

динамическая смена окружения

%наличие карты глубины окружения

работа с динамическими тенями

\section{Классификация методов}



\section*{Вывод}

В данном разделе был проведен обзор существующих методов наложения теней в ДР, введены критерии сравнения и проведена классификация этих методов.