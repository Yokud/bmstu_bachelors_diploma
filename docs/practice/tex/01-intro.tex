\chapter*{ВВЕДЕНИЕ}
\addcontentsline{toc}{chapter}{ВВЕДЕНИЕ}

Во время выполнения выпускной квалификационной работы был разработан метод наложение теней в дополненной реальности на основе информации о глубине точек кадра. 

Зачастую методы наложения теней в дополненной реальности вычисляют положения ИС в каждом кадре, что требует немало вычислительных ресурсов \cite{osti2019real}. В тех случаях, когда схема освещения окружения в основном не изменятся, это избыточно. Разработанный метод предназначен для таких случаев и требует вычислить положения ИС только в начале сессии или при необходимости, когда схема освещения коренным образом изменилась, что позволяет сократить использование вычислительных ресурсов, тратя их только на обработку компьютерной графики и построение геометрии окружения.

\textbf{Цель работы} -- реализовать программное обеспечение, демонстрирующее практическую осуществимость спроектированного в ходе выполнения выпускной квалификационной работы метода.

Для достижения поставленной цели нужно решить следующие задачи:

\begin{itemize}
	\item[---] спроектировать структуру программного обеспечения, реализующего разработанный метод;
	\item[---] разработать программное обеспечение для данного метода;
	\item[---] продемонстрировать работоспособность реализованного программного обеспечения.
\end{itemize}