\chapter{Основная часть}

\section{Формальная постановка задачи}

На рисунке \ref{img:01_A0} представлена формализация задачи в виде IDEF0-диаграммы.

\includeimage
{01_A0}
{f}
{H}
{\textwidth}
{Формализация задачи в виде IDEF0-диаграммы}

\section{Этапы метода}

На рисунке \ref{img:02_A0} представлена IDEF0-диаграмма описанного метода.

\includeimage
{02_A0}
{f}
{H}
{\textwidth}
{IDEF0-диаграмма описанного метода}

На рисунках 1.3-1.7 представлены схемы этапов разработанного метода.

\includeimage
{ProcessingImage.drawio}
{f}
{H}
{0.8\textwidth}
{Схема алгоритма обработки изображения}

\includeimage
{SearchingLightSources.drawio}
{f}
{H}
{0.27\textwidth}
{Схема алгоритма поиска положения источников света}

\includeimage
{CameraOrientation.drawio}
{f}
{H}
{0.325\textwidth}
{Схема алгоритма определения ориентации камеры в пространстве}

\includeimage
{MeshBuilding.drawio}
{f}
{H}
{0.325\textwidth}
{Схема алгоритма построения геометрии окружения в реальном времени}

\includeimage
{ShadowDrawing.drawio}
{f}
{H}
{0.25\textwidth}
{Схема алгоритма отрисовки теней}

\section{Ограничения метода}

Разработанный метод имеет следующие ограничения:

\begin{itemize}
	\item[---] метод применим только для работы в помещении;
	\item[---] полигональная сетка окружения не учитывает свойства поверхности, такие как отражение поверхностью света и прозрачность.
\end{itemize}

Первое ограничение связано с тем, что на улице сложно однозначно определить положение источников света из-за рассеянново света \cite{osti2019real}. 

\subsection{Ограничения на входные и выходные данные}

В таблице \ref{FormatData} приведено описание формата входных данных.

\begin{table}[H]
	\caption{Описание входных данных}
	\label{FormatData}
	\begin{center}
		\begin{tabular}{| p{6 cm} | p{9 cm} |} 
			\hline
			Данные & Формат \\
			\hline
			Сферическая панорама окружения & Цветное изображение формата PNG с глубиной цвета 32 бит \\
			\hline
			Сферическая панорама глубины окружения & Изображение в оттенках серого формата PNG с глубиной цвета 64 бит \\
			\hline
			Виртуальный трехмерный объект & Файл формата prefab \\
			\hline
			Кадр окружения & Цветное изображение \\
			\hline
			Кадр глубины окружения & Изображение в оттенках серого \\
			\hline
		\end{tabular}
	\end{center}
\end{table}

Ограничения на входные данные:

\begin{itemize}
	\item[---] обе сферические панорамы должны быть с одинаковым разрешением;
	\item[---] оба кадра окружения должны быть с одинаковым разрешением;
	\item[---] минимальное расстояние от точки съемки окружения должно соответствовать минимальному расстоянию распознавания глубины используемой 3D-камеры;
	\item[---] максимальное расстояние от точки съемки окружения должно соответствовать максимальному расстоянию распознавания глубины используемой 3D-камеры.
\end{itemize}

Выходными данными является цветное изображение, разрешение которого соответсвует разрешению кадров окружения.

\section{Разработанное программное обеспечение}

Разработанное программного обеспечения представляет собой проект Unity \cite{unity}, который состоит из набора сценариев на языке C\# \cite{c-sharp}. В качестве 3D-камеры используется Kinect \cite{kinect}.

\subsection{Требования к вычислительной системе}

Для работы реализованного программного обеспечения требуется:

\begin{itemize}
	\item[---] 3D-камера Kinect и Kinect SDK версии 1.8;
	\item[---] Поддержка Visual C++ 2015;
	\item[---] Поддержка .NET Standart 2.1.
\end{itemize}

Для работы с проектом реализованного программного обеспечения требуется Unity версии 2021.3.23f1.

\subsection{Основные модули}

Далее представлены основные модули разработанного программного обеспечения:

\begin{itemize}
	\item[---] LightPosCalc -- предназначен для вычисления положения источников света;
	\item[---] KinectManager -- предназначен для взаимодействия с Kinect (получения кадра окружения и кадра глубины окружения);
	\item[---] ExamplesManager -- предназначен для работы с набором доступных моделей.
\end{itemize}

\subsection{Руководство пользователя}

В разработанном программном обеспечении существует система управления трехмерными моделями. После загрузки панорма и вычисления положения источников света загружается сцена, где пользователю доступно создание, удаление и управление положением и поворотом заранее предопределенного набора моделей, представленные на рисунке \ref{img:models}. 

\includeimage
{models}
{f}
{H}
{0.5\textwidth}
{Набор доступных моделей}

\subsubsection*{Создание моделей}

Создание моделей производится посредством наведения курсора на некоторую часть экрана и нажатием клавиши на клавиатуре \textbf{1}, \textbf{2} или \textbf{3}, что добавляет на сцену модели параллелепипеда, кота и пальмы соответственно. Они создаются от камеры на некотором расстоянии, значение которого задается конфигурационным файлом.

\subsubsection*{Удаление моделей}

Удаление моделей происходит с помощью клавиши \textbf{Delete}: нужно навести курсор на некоторую модель на сцене, нажать ЛКМ и нашать клавишу \textbf{Delete}.

\subsubsection*{Управление моделями}

Для управления моделью нужно нажать ЛКМ на некоторую модель на сцене, после чего пользователю доступно:

\begin{itemize}
	\item[---] перемещение модели вдоль мировой оси X (влево-право) клавишами \textbf{A} и \textbf{D} соответственно;
	\item[---] перемещение модели вдоль мировой оси Z (вперед-назад) клавишами \textbf{W} и \textbf{S} соответственно;
	\item[---] перемещение модели вдоль мировой оси Y (вверх-вниз) клавишами \textbf{Space} и \textbf{Shift} соответственно.
	\item[---] поворот модели вокруг собственной оси X по часовой стрелке и против часовой стрелки клавишами \textbf{R} и \textbf{F} соответственно;
	\item[---] поворот модели вокруг собственной оси Y по часовой стрелке и против часовой стрелки клавишами \textbf{T} и \textbf{G} соответственно;
	\item[---] поворот модели вокруг собственной оси Z по часовой стрелке и против часовой стрелки клавишами \textbf{Y} и \textbf{H} соответственно.
\end{itemize}

\subsection{Примеры работы}

На рисунке \ref{img:UI} представлен графический интерфейс разработанного программного обеспечения.

\includeimage
{UI}
{f}
{H}
{0.5\textwidth}
{Графический интерфейс разработанного программного обеспечени}

На рисунке \ref{img:work_example} представлен пример работы разработанного программного обеспечения.

\includeimage
{work_example}
{f}
{H}
{0.5\textwidth}
{Пример работы}