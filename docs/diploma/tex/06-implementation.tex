\chapter{Технологический раздел}

В данном разделе обосновывается выбор языка и средств программной реализации метода наложения теней в дополненной реальности; описывается разработанное программное обеспечение, реализующее описанный метод, и проводится его тестирование. Также описывается формат входных и выходных данных и взаимодействие пользователя с программным обеспечением.

\section{Выбор языка и средств программной реализации метода}

Для обработки ИС, виртуальных теней, трехмерных моделей и прочих элементов машинной графики был выбран игровой движок Unity \cite{unity}, т. к. он кроссплатформенный, обладает бесплатной версией, функционала которого достаточно для реализции метода, обширной документацией и множеством обучающего материала, что позволяет легко его освоить. 

В качестве языка программирования был выбраны C\# \cite{c-sharp}, т. к. скрипты для Unity используют этот язык. Также этот язык обладает удобным синтаксисом, управляемым кодом и сборщиком мусора, благодаря этому не нужно заботится об утечках памяти, об указателях и о некоторых базовых структурах и алгоритмах -- все это уже реализовано. Это позволит ускорить разработку и отладку кода.

В качестве среды разработки (IDE) была выбрана Visual Studio \cite{vs}, обладающая интеллектуальными подсказками, инструментами анализа, отладки и тестирования кода, поставляющаяся вместе с языком C\# и пакетным менеджером NuGet \cite{nuget}, который позволяет в случае чего дополнить недостающий функционал языка или игрового движка.