\chapter{Технологический раздел}

В данном разделе обосновывается выбор языка и средств программной реализации метода наложения теней в дополненной реальности; описывается разработанное программное обеспечение, реализующее описанный метод, и проводится его тестирование. Также описывается формат входных и выходных данных и взаимодействие пользователя с программным обеспечением.

\section{Выбор языка и средств программной реализации метода}

Для обработки ИС, виртуальных теней, трехмерных моделей и прочих элементов машинной графики был выбран игровой движок Unity \cite{unity}, т. к. он кроссплатформенный, обладает бесплатной версией, функционала которого достаточно для реализции метода, обширной документацией и множеством обучающего материала, что позволяет легко его освоить. 

В качестве языка программирования был выбраны C\# \cite{c-sharp}, т. к. скрипты для Unity используют этот язык. Также этот язык обладает удобным синтаксисом, управляемым кодом и сборщиком мусора, благодаря этому не нужно заботится об утечках памяти, об указателях и о некоторых базовых структурах и алгоритмах -- все это уже реализовано. Это позволит ускорить разработку и отладку кода.

В качестве среды разработки (IDE) была выбрана Visual Studio \cite{vs}, обладающая интеллектуальными подсказками, инструментами анализа, отладки и тестирования кода, поставляющаяся вместе с языком C\# и пакетным менеджером NuGet \cite{nuget}, который позволяет в случае чего дополнить недостающий функционал языка или игрового движка.

Для анализа и обработки изображений (анализ гистограммы, поиск контуров и т. д.) была выбрана библиотека OpenCV \cite{opencv}, а точнее обертка над ней OpenCVSharp \cite{opencvsharp}, учитывая ранее выбранный язык. Для захвата данных об окружении была выбрана 3D-камера Kinect \cite{kinect}, т. к. к ней прилагается ПО, которое позволяет использовать ее вместе с Unity.

\section{Ограничения реализации метода}

Т. к. в реализации метода используется Kinect, то ограничения накладываются от используемой 3D-камеры:

\begin{itemize}
	\item[---] минимальное расстояние от точки съемки окружения должно быть не менее 0.8 метров;
	\item[---] максимальное расстояние от точки съемки окружения должно быть не более 4 метров.
\end{itemize}

Соблюдая эти условия, Kinect способен захватить информацию о глубине окружения. В ином случае значения глубины, полученные с Kinect, будут равны нулю.

\section{Демонстрация метода}

На рисунке \ref{} представлено сравнение поведения теней от реального и виртуального объекта.



Как видно из результата сравнения, поведение тени от виртуального объекта и реального совпадают.

\section{Формат входных и выходных данных}

В таблице \ref{FormatData} приведено описание формата входных данных.

\begin{table}[H]
	\caption{Описание входных данных}
	\label{FormatData}
	\begin{center}
		\begin{tabular}{| p{6 cm} | p{9 cm} |} 
			\hline
			Данные & Формат \\
			\hline
			Сферическая панорама окружения & Цветное изображение формата PNG \\
			\hline
			Набор изображений, из которых составляется сферическая панорама окружения & Цветные изображения формата PNG \\
			\hline
			Сферическая панорама глубины окружения & Изображение в оттенках серого формата PNG \\
			\hline
			Набор изображений, из которых составляется сферическая панорама глубины окружения & Изображения в оттенках серого формата PNG \\
			\hline
		\end{tabular}
	\end{center}
\end{table}

Выходными данными является цветное изображение с разрешением 640 на 480 пикселей.

\section{Взаимодействие пользователя с программным обеспечением}

На рисунке \ref{img:menu} представлен начальный экран реализованного ПО.

\includeimage
{menu}
{f}
{H}
{0.625\textwidth}
{Начальный экран реализованного ПО}

Для загрузки данных об окружении пользователю необходимо загрузить сферическую панораму окружения или директорию с набором изображений, из которых можно составить сферическую панораму, и сферическую панораму глубины окружения или директорию с набором изображений, из которых можно составить сферическую панораму глубины, для чего на начальном экране присутствуют соответствующие кнопки.

Для демонстрации метода присутствует кнопка <<Демонстрация метода>>, которая загружает сцену, где демострируется восстановленная система освещения и взаимодействие виртуального объекта с реальными. Эта кнопка не будет действовать до тех пор, пока не будут загружены данные об окружении и данные о глубине окружения. 

На рисунке \ref{img:test_diploma} представлен пример работы реализованного метода.

\includeimage
{test_diploma}
{f}
{H}
{0.625\textwidth}
{Пример работы реализованного метода}

В верхнем правом углу отображается статус работы Kinect. Если при взаимодействии с Kinect возникла ошибка, то в этом поле отображается текст ошибки.

Т. к. у Kinect нет GPS-модуля и гироскопа, то синхронизировать положение виртуальной камеры с положением Kinect в помещении требуется пользователю. Для этого в верхнем левом углу отображаются данные о позиции и повороте виртуальной камеры. Значения координат отображаются в сантиметрах, углы поворота -- в градусах. Чтобы изменить координаты виртуальной камеры, используются следующие клавиши:

\begin{itemize}
	\item[---] W -- увеличивает значение координаты Z;
	\item[---] A -- увеличивает значение координаты X;
	\item[---] S -- уменьшает значение координаты Z;
	\item[---] D -- уменьшает значение координаты X;
	\item[---] Space -- увеличивает значение координаты Y;
	\item[---] Shift -- уменьшает значение координаты Y.
\end{itemize}

Для изменения поворота виртуальной камеры используется компьютерная мышь.

Нулевая точка соответствует точке, в которой происходила съемка панорам окружения.

\section*{Вывод}

В данном разделе был обоснован выбор языка и средств программной реализации метода наложения теней в дополненной реальности; описано разработанное программное обеспечение, реализующее описанный метод, и проведено его тестирование. Также был описан формат входных и выходных данных и взаимодействие пользователя с программным обеспечением.