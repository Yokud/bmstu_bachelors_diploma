\chapter{Технологический раздел}

В данном разделе обосновывается выбор языка и средств программной реализации метода наложения теней в дополненной реальности; описывается разработанное программное обеспечение, реализующее описанный метод, и проводится его тестирование. Также описывается формат входных и выходных данных и взаимодействие пользователя с программным обеспечением.

\section{Выбор языка и средств программной реализации метода}

Для обработки ИС, виртуальных теней, трехмерных моделей и прочих элементов машинной графики был выбран игровой движок Unity \cite{unity}, поскольку он кроссплатформенный, обладает бесплатной версией, функционала которого достаточно для реализции метода, обширной документацией и множеством обучающего материала, что позволяет легко его освоить. 

В качестве языка программирования был выбраны C\# \cite{c-sharp}, поскольку скрипты для Unity используют этот язык. Также этот язык обладает удобным синтаксисом, управляемым кодом и сборщиком мусора, благодаря этому не нужно заботится об утечках памяти, об указателях и о некоторых базовых структурах и алгоритмах -- все это уже реализовано. Это позволит ускорить разработку и отладку кода.

В качестве среды разработки (IDE) была выбрана Visual Studio \cite{vs}, обладающая интеллектуальными подсказками, инструментами анализа, отладки и тестирования кода, поставляющаяся вместе с языком C\# и пакетным менеджером NuGet \cite{nuget}, который позволяет в случае чего дополнить недостающий функционал языка или игрового движка.

Для анализа и обработки изображений (анализ гистограммы, поиск контуров и т. д.) была выбрана библиотека OpenCV \cite{opencv}, а точнее обертка над ней OpenCVSharp \cite{opencvsharp}, учитывая ранее выбранный язык. Для захвата данных об окружении была выбрана 3D-камера Kinect \cite{kinect}, поскольку к ней прилагается ПО, которое позволяет использовать ее вместе с Unity.

\section{Ограничения метода}

Так как в реализации метода используется Kinect, ограничения накладываются используемой 3D-камерой:

\begin{itemize}
	\item[---] минимальное расстояние от точки съемки окружения должно быть не менее 0.8 метров;
	\item[---] максимальное расстояние от точки съемки окружения должно быть не более 4 метров.
\end{itemize}

Соблюдая эти условия, Kinect способен захватить информацию о глубине окружения. В ином случае значения глубины, полученные с Kinect, будут равны нулю.

Также стоит учитывать следующие ограничения метода:

\begin{itemize}
	\item[---] частота изменения системы освещения минимальна или равна нулю;
	\item[---] отношение максимального значения яркости к среднему значению яркости сферической панорамы окружения должно быть не менее 1.5;
	\item[---] любой ИС на сферической панораме окружения интерпретируется как точечный ИС с белым свечением;
	\item[---] полигональная сетка окружения не учитывает свойства поверхности, такие как прозрачность и альбедо.
\end{itemize}

\section{Формат входных и выходных данных}

В таблице \ref{FormatData} приведено описание формата входных данных.

\begin{table}[H]
	\caption{Описание входных данных}
	\label{FormatData}
	\begin{center}
		\begin{tabular}{| p{6 cm} | p{9 cm} |} 
			\hline
			Данные & Формат \\
			\hline
			Сферическая панорама окружения & Цветное изображение формата PNG с глубиной цвета 32 бит \\
			\hline
			Сферическая панорама глубины окружения & Изображение в оттенках серого формата PNG с глубиной цвета 64 бит \\
			\hline
			Виртуальный трехмерный объект & Файл формата prefab \\
			\hline
			Кадр окружения & Цветное изображение с разрешением 640 на 480 пикселей \\
			\hline
			Кадр глубины окружения & Изображение в оттенках серого с разрешением 640 на 480 пикселей \\
			\hline
		\end{tabular}
	\end{center}
\end{table}

Ограничения на входные данные:

\begin{itemize}
	\item[---] обе сферические панорамы должны быть с одинаковым разрешением;
	\item[---] оба кадра окружения должны быть с одинаковым разрешением.
\end{itemize}

Выходными данными является цветное изображение с разрешением 640 на 480 пикселей.

\section{Описание сценариев Unity}

Сценарий Unity представляет собой файл с расширением \textbf{.cs}, в котором написан некоторый код на языке C\#.

В разработанном программном обеспечении были реализованы следующие сценарии Unity:

\begin{itemize}
	\item[---] SavePano, SaveDepthPano предназначены для загрузки сферической панорамы окружения, загрузки сферической панорамы глубины окружения соответственно;
	\item[---] EnvDataFields предназначен для хранения загруженных панорам и координат найденных ИС;
	\item[---] LightPosCalc представляет собой алгоритм вычисления положения ИС;
	\item[---] KinectManager и KinectWrapper предназначены для работы с Kinect, а также KinectManager отвечает за построение полигональной сетки окружения;
	\item[---] SceneManager отвечает за расстановку ИС по найденным координатам, а также за выход в меню;
	\item[---] ExamplesManager и ExampleManager отвечают за управление моделями на сцене;
	\item[---] CameraMove и CameraRotate предназначены для перемещения и поворота виртуальной камеры соответственно.
\end{itemize}

\section{Демонстрация метода}

На рисунке \ref{img:work_test} представлено сравнение поведения теней от реального и виртуального объекта.

\includeimage
{work_test}
{f}
{H}
{0.625\textwidth}
{Сравнение поведения теней от реального и виртуального объекта}

Как видно из результата сравнения, направления теней от виртуального объекта и реального совпадают, даже наблюдается искажение тени при проецировании на вертикальную и горизонтальную поверхность, различия лишь в интенсивности тени у виртуального объекта: она меньше, чем у тени от реального объекта.

Рябь изображения объясняется шумовым характером данных о глубине с Kinect.

\section{Взаимодействие пользователя с программным обеспечением}

На рисунке \ref{img:menu} представлен начальный экран реализованного ПО.

\includeimage
{menu}
{f}
{H}
{0.625\textwidth}
{Начальный экран реализованного ПО}

Для загрузки данных об окружении пользователю необходимо загрузить сферическую панораму окружения или директорию с набором изображений, из которых можно составить сферическую панораму, и сферическую панораму глубины окружения или директорию с набором изображений, из которых можно составить сферическую панораму глубины, для чего на начальном экране присутствуют соответствующие кнопки.

Для демонстрации метода присутствует кнопка <<Демонстрация метода>>, которая загружает сцену, где демострируется восстановленная система освещения и взаимодействие виртуального объекта с реальными. Эта кнопка не будет действовать до тех пор, пока не будут загружены данные об окружении и данные о глубине окружения. 

Для возврата в меню нужно нажать клавишу \textbf{Escape}.

\subsection{Управление виртуальной камерой}

В верхнем правом углу отображается статус работы Kinect. Если при взаимодействии с Kinect возникла ошибка, то в этом поле отображается текст ошибки. Чуть ниже в том же углу отображается статус расставления ИС: если возникнет ошибка, в этом окне появится об этом информация.

Т. к. у Kinect нет GPS-модуля и гироскопа, то синхронизировать положение виртуальной камеры с положением Kinect в помещении требуется пользователю. Для этого в верхнем левом углу отображаются данные о позиции и повороте виртуальной камеры. Значения координат отображаются в сантиметрах, углы поворота -- в градусах. Чтобы изменить координаты виртуальной камеры, используются следующие клавиши:

\begin{itemize}
	\item[---] W -- увеличивает значение координаты Z;
	\item[---] A -- увеличивает значение координаты X;
	\item[---] S -- уменьшает значение координаты Z;
	\item[---] D -- уменьшает значение координаты X;
	\item[---] Space -- увеличивает значение координаты Y;
	\item[---] Shift -- уменьшает значение координаты Y.
\end{itemize}

Для изменения поворота виртуальной камеры используются следующие клавиши:

\begin{itemize}
	\item[---] \textbf{R} и \textbf{F} для поворота вверх и вниз соответственно;
	\item[---] \textbf{T} и \textbf{G} для поворота влево и вправо соответственно.
\end{itemize}

Нулевая точка соответствует точке, в которой происходила съемка панорам окружения. Нулевые значения поворота камеры соответствуют направлению в центре панорамы.

\subsection{Управление моделями на сцене}

В разработанном программном обеспечении существует система управления трехмерными моделями. После загрузки панорам и вычисления положения источников света загружается сцена, где пользователю доступно создание, удаление и управление положением и поворотом заранее предопределенного набора моделей, представленные на рисунке \ref{img:models}. 

\includeimage
{models}
{f}
{H}
{0.5\textwidth}
{Набор доступных моделей}

\subsubsection*{Создание моделей}

Создание моделей производится посредством наведения курсора на некоторую часть экрана и нажатием клавиши на клавиатуре \textbf{1}, \textbf{2} или \textbf{3}, что добавляет на сцену модели параллелепипеда, кота и пальмы соответственно. Они создаются от камеры на некотором расстоянии, значение которого задается конфигурационным файлом.

\subsubsection*{Выбор модели на сцене}

Выбор модели, имеющейся на сцене, осуществляется с помощью ЛКМ: нужно навести курсор на некоторую модель на сцене и нажать ЛКМ. 

Чтобы сбросить выбранную модель, достаточно нажать ПКМ в любой части экрана.

\subsubsection*{Удаление моделей}

Удаление моделей происходит с помощью клавиши \textbf{Delete}: нужно выбрать модель на сцене и после нашать клавишу \textbf{Delete}.

\subsubsection*{Управление моделями}

Для управления выбранной моделью на сцене пользователю доступно:

\begin{itemize}
	\item[---] перемещение модели вдоль мировой оси X (влево-право) клавишами \textbf{A} и \textbf{D} соответственно;
	\item[---] перемещение модели вдоль мировой оси Z (вперед-назад) клавишами \textbf{W} и \textbf{S} соответственно;
	\item[---] перемещение модели вдоль мировой оси Y (вверх-вниз) клавишами \textbf{Space} и \textbf{Shift} соответственно.
	\item[---] поворот модели вокруг собственной оси X по часовой стрелке и против часовой стрелки клавишами \textbf{R} и \textbf{F} соответственно;
	\item[---] поворот модели вокруг собственной оси Y по часовой стрелке и против часовой стрелки клавишами \textbf{T} и \textbf{G} соответственно;
	\item[---] поворот модели вокруг собственной оси Z по часовой стрелке и против часовой стрелки клавишами \textbf{Y} и \textbf{H} соответственно.
\end{itemize}

\section*{Вывод}

В данном разделе был обоснован выбор языка и средств программной реализации метода наложения теней в дополненной реальности; описаны ограничения метода и формат входных и выходных данных. Также было продемострировано разработанное программное обеспечение, реализующее описанный метод, проведена проверка его работоспособсности и описано взаимодействие пользователя с программным обеспечением.