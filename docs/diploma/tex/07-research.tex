\chapter{Исследовательский раздел}

\section{Классификация поверхностей}

С учетом ограничений метода, была предложена следующая классификация поверхностей:

\begin{itemize}
	\item[---] плоские поверхности -- на таких поверхностях искажение тени будет минимальным или отсутствовать вовсе;
	\item[---] неровные поверхности с высотными различиями, текстурой или рельефом -- на таких поверхностях искажение тени может быть заметным, если есть выступы или углубления;
	\item[---] поверхности с наличием объектов или препятствий -- на таких поверхностях искажение тени может быть заметным и зависеть от положения объектов или препятствий на пути света.
\end{itemize}

\section{Методика проведения исследования}

Для исследования необходимо произвести несколько снимков с виртуальным объектом с проецированием тени от него на все типы поверхности, указанные выше. Также нужно ввести экспертную оценку для оценивания степени реалистичности искажения тени от виртуального объекта.

На рисунках \ref{img:test11}-\ref{img:test13} представлены снимки с демонстрацией искажения тени от виртуального объекта для плоской поверхности.

\includeimage
{test11}
{f}
{H}
{0.4\textwidth}
{Поведение тени от виртуального объекта для плоской поверхности, пример 1}

\includeimage
{test12}
{f}
{H}
{0.4\textwidth}
{Поведение тени от виртуального объекта для плоской поверхности, пример 2}

\includeimage
{test13}
{f}
{H}
{0.4\textwidth}
{Поведение тени от виртуального объекта для плоской поверхности, пример 3}

На рисунках \ref{img:test21}-\ref{img:test23} представлены снимки с демонстрацией искажения тени от виртуального объекта для неровной поверхности.

\includeimage
{test21}
{f}
{H}
{0.4\textwidth}
{Поведение тени от виртуального объекта для неровной поверхности, пример 1}

\includeimage
{test22}
{f}
{H}
{0.4\textwidth}
{Поведение тени от виртуального объекта для неровной поверхности, пример 2}

\includeimage
{test23}
{f}
{H}
{0.4\textwidth}
{Поведение тени от виртуального объекта для неровной поверхности, пример 3}

На рисунках \ref{img:test31}-\ref{img:test33} представлены снимки с демонстрацией искажения тени от виртуального объекта для поверхности с наличием объектов или препятствий.

\includeimage
{test31}
{f}
{H}
{0.4\textwidth}
{Поведение тени от виртуального объекта для поверхности с наличием объектов или препятствий, пример 1}

\includeimage
{test32}
{f}
{H}
{0.4\textwidth}
{Поведение тени от виртуального объекта для поверхности с наличием объектов или препятствий, пример 2}

\includeimage
{test33}
{f}
{H}
{0.4\textwidth}
{Поведение тени от виртуального объекта для поверхности с наличием объектов или препятствий, пример 3}

На основе предложенных снимков была введена пятибальная шкала оценки реалистичности формируемой тени, где 1 -- поведение тени совершенно не похоже на поведение реальной тени и 5 -- поведение тени полностью соответствует поведению реальной тени. Также был проведен опрос 10 человек в целях получения экспертной оценки степени реалистичности искажения тени от виртуального объекта, результаты которого представлены в таблице \ref{Otsenka}.

\begin{table}[H]
	\caption{Экспертная оценка}
	\label{Otsenka}
	\begin{center}
		\begin{tabular}{| p{8 cm} | p{3.5 cm} |} 
			\hline
			Тип поверхности & Средняя оценка \\
			\hline
			Плоская поверхность & 4.3 \\
			\hline
			Неровная поверхность & 4.1 \\
			\hline
			Поверхность с наличием объектов или препятствий & 3.9 \\
			\hline
		\end{tabular}
	\end{center}
\end{table}

Из результатов оценки видно, что в реализованном методе поведение тени от виртуального объекта на различные поверхности соответствует реальной тени, особенно для плоских и неровных поверхностей.

\section{Сравнение с аналогами}

Сравнение методов проводилось на ПК с техническими характеристиками:

\begin{itemize}
	\item[---] ЦПУ: Intel Core i7 4790K;
	\item[---] ОЗУ: 16 Гб DDR3;
	\item[---] ГПУ: Nvidia RTX 3070;
	\item[---] ОС: Windows 10.
\end{itemize}

Для разработанного метода проводилось 1000 замеров времени детектирования ИС, после чего было выбрано среднее значение детектирования ИС.

Для остальных методов замеры были взяты из статей, где они были описаны, и аппроксимированы с учетом разницы технических характеристик из-за отсутствия исходного кода для методов, описанных в статьях \cite{shadow_contours_method} \cite{THOMASIAN2022385} \cite{sns_tras}.

В таблице \ref{CompareMethods} представлены результаты сравнения методов.

\begin{table}[H]
	\caption{Результаты сравнения методов}
	\label{CompareMethods}
	\begin{center}
		\begin{tabular}{| p{8 cm} | p{3.5 cm} |} 
			\hline
			Метод & Детектирование ИС, мс \\
			\hline
			Метод на основе анализа контуров теней & 4 \\
			\hline
			Метод на основе построения теневых объемов & 97 \\
			\hline
			Метод с использованием сверточных нейронных сетей и трассировки теневых лучей & 932 \\
			\hline
			Разработанный метод & 215 \\
			\hline
		\end{tabular}
	\end{center}
\end{table}

\section{Вывод}

В данном разделе было проведено исследование результатов разработанного метода при проецировании теней от виртуального объекта на различные поверхности. Также было проведено сравнение разработанного метода с существующими аналогами.

В результате исследования было установлено, что в реализованном методе поведение тени от виртуального объекта на различные поверхности соответствует реальной тени, особенно для плоских и неровных поверхностей.

В результате сравнения методов было установлено, что время расчета обнаружения ИС у разработанного метода больше, чем у метода на основе анализа контуров теней и метода на основе построения теневых объемов, но стоит учесть, что в разработанном методе процедура детектирования ИС выполняется единожды, а не в каждом кадре, как в других методах, что позволяет использовать больше вычислительных ресурсов на качество обработки компьютерной графики.