\chapter*{РЕФЕРАТ}
\addcontentsline{toc}{chapter}{РЕФЕРАТ}

Объем РПЗ составляет \begin{NoHyper}\pageref{LastPage}\end{NoHyper} страниц, содержит \totfig~иллюстрации, \tottab~таблицы, и 32 использованных источников.

Объектом разработки является метод наложения теней в дополненной реальности на основе информации о глубине точек кадра.

\textbf{Цель работы} -- разработать метод наложения теней в дополненной реальности.

Для достижения поставленной цели нужно решить следующие задачи:

\begin{itemize}
	\item[---] провести анализ предметной области наложения теней;
	\item[---] провести обзор существующих методов наложения теней в ДР на основе информации о глубине точек кадра и привести результаты сравнительного анализа;
	\item[---] разработать и описать собственный метод наложения теней в ДР на основе информации о глубине точек кадра, который будет вычислять положения ИС только в начале сессии или при необходимости;
	\item[---] разработать программное обеспечение, реализующее описанный метод, и проверить его работоспособность;
	\item[---] провести исследование результатов разработанного метода при проецировании теней от виртуального объекта на различные поверхности;
	\item[---] выполнить сравнение результатов работы реализованного метода с результатами, полученными с помощью существующих аналогов.
\end{itemize}

Поставленная цель была достигнута: в ходе выполнения выпускной квалификационной работы был разработан метод наложения теней в дополненной реальности на основе информации о глубине точек кадра. Разработанный метод позволяет учитывать форму поверхности окружения при проецировании тени от виртуального объекта. Также он учитывает данные глубины при рассчете положения ИС.

\textit{Ключевые слова}: дополненная реальность, компьютерное зрение, компьютерная графика, наложение теней, RGBD-изображение, восстановление параметров освещения, глубина точек кадра, карта глубины кадра, сферическая панорама.