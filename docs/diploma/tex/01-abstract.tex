\chapter*{РЕФЕРАТ}
\addcontentsline{toc}{chapter}{РЕФЕРАТ}

Объем РПЗ составляет \begin{NoHyper}\pageref{LastPage}\end{NoHyper} страницу, содержит \totfig~иллюстрации, \tottab~таблиц, и 32 использованных источника.

Объектом разработки является метод наложения теней в дополненной реальности на основе информации о глубине точек кадра.

Проведен анализ предметной области наложения теней, обзор существующих методов наложения теней в дополненной реальности на основе информации о глубине точек кадра, и приведены результаты сравнительного анализа. Разработан и описан собственный метод наложения теней в дополненной реальности на основе информации о глубине точек кадра, который будет вычислять положения источников света только в начале сессии или при необходимости. Разработано программное обеспечение, реализующее описанный метод, и проверена его работоспособность. Проведено исследование результатов разработанного метода при проецировании теней от виртуального объекта на различные поверхности и выполнено сравнение результатов работы реализованного метода с результатами, полученными с помощью существующих аналогов.

\textit{Ключевые слова}: дополненная реальность, компьютерное зрение, компьютерная графика, наложение теней, RGBD-изображение, восстановление параметров освещения, глубина точек кадра, карта глубины кадра, сферическая панорама.