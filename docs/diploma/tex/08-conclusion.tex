\chapter*{ЗАКЛЮЧЕНИЕ}
\addcontentsline{toc}{chapter}{ЗАКЛЮЧЕНИЕ}

В ходе выполнения выпускной квалификационной работы был разработан метод наложения теней в дополненной реальности на основе информации о глубине точек кадра.

Для достижения поставлен­ной цели были решены следующие задачи:

\begin{itemize}
	\item[---] проведен анализ предметной области наложения теней;
	\item[---] проведен обзор существующих методов наложения теней в ДР на основе информации о глубине точек кадра и привести результаты сравнительного анализа;
	\item[---] разработан и описан собственный метод наложения теней в ДР на основе информации о глубине точек кадра, который будет вычислять положения ИС только в начале сессии или при необходимости;
	\item[---] разработано программное обеспечение, реализующее описанный метод, и выполнить его тестирование;
	\item[---] проведено исследование результатов разработанного метода при проецировании теней от виртуального объекта на различные поверхности;
	\item[---] выполнено сравнение результатов работы реализованного метода с результатами, полученными с помощью существующих аналогов.
\end{itemize}

Разработанный метод позволяет учитывать форму поверхности окружения при проецировании тени от виртуального объекта. Также он учитывает данные глубины при рассчете положения ИС.

В качестве развития проекта было предложено следующее:

\begin{itemize}
	\item[---] реализация автоматического определения ориентации камеры в пространстве;
	\item[---] определение типа ИС по характеру свечения (точечный, направленный и т. д.).
\end{itemize}