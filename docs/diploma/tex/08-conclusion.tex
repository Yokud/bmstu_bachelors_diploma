\chapter*{ЗАКЛЮЧЕНИЕ}
\addcontentsline{toc}{chapter}{ЗАКЛЮЧЕНИЕ}

В ходе выполнения выпускной квалификационной работы был разработан метод наложения теней в дополненной реальности на основе информации о глубине точек кадра.

Был проведен анализ предметной области наложения теней, обзор существующих методов наложения теней в ДР на основе информации о глубине точек кадра, и были приведены результаты сравнительного анализа.

Был разработан и описан собственный метод наложения теней в ДР на основе информации о глубине точек кадра, который будет вычислять положения ИС только в начале сессии или при необходимости. Для реализции метода было разработанно программное обеспечение и проверена его работоспособность.

Разработанный метод позволяет учитывать форму поверхности окружения при проецировании тени от виртуального объекта. Также он учитывает данные глубины при рассчете положения ИС.

Было проведено исследование результатов разработанного метода при проецировании теней от виртуального объекта на различные поверхности, а также выполнено сравнение временных затрат на поиск источников света реализованного метода с результатами, полученными с помощью существующих аналогов.

В качестве развития проекта было предложено следующее:

\begin{itemize}
	\item[---] реализация автоматического определения ориентации камеры в пространстве;
	\item[---] определение типа ИС по характеру свечения (точечный, направленный и т. д.).
\end{itemize}